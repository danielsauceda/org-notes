% Created 2015-12-01 mar 08:31
\documentclass[11pt]{article}
\usepackage[utf8]{inputenc}
\usepackage[T1]{fontenc}
\usepackage{fixltx2e}
\usepackage{graphicx}
\usepackage{longtable}
\usepackage{float}
\usepackage{wrapfig}
\usepackage{soul}
\usepackage{textcomp}
\usepackage{marvosym}
\usepackage{wasysym}
\usepackage{latexsym}
\usepackage{amssymb}
\usepackage{hyperref}
\tolerance=1000
\providecommand{\alert}[1]{\textbf{#1}}

\title{User-concerns:MDCS}
\author{danielsauceda}
\date{\today}
\hypersetup{
  pdfkeywords={},
  pdfsubject={},
  pdfcreator={Emacs Org-mode version 7.9.3f}}

\begin{document}

\maketitle

\setcounter{tocdepth}{3}
\tableofcontents
\vspace*{1cm}


\section{Bugs , Questions, and concerns}
\label{sec-1}
\subsection{Questions}
\label{sec-1-1}
\subsubsection{widgets and scripts}
\label{sec-1-1-1}

Is it possible to use incorporate UI made schema with widgets via scripts.
\subsection{Bugs}
\label{sec-1-2}
\subsubsection{Faulty Caching}
\label{sec-1-2-1}

\begin{itemize}
\item \textbf{First}) We create a \verb~user~ defined schema with composer
\item \textbf{Second}) We enter data using the WEB-UI
\item \textbf{Third}) We enter in an \underline{incorrect} value into a field (\emph{on purpose})
\item \textbf{Fourth}) we attempt to upload the data
\item \textbf{RESULT}) ERROR showing an incorrect value
\item \textbf{BUG}) Despite changing the the value; system maintains error.
\item \textbf{Pseudo-FIX}) Uninstalling entire system
\end{itemize}
\subsubsection{USER DEFINED schemas API upload failure}
\label{sec-1-2-2}

\begin{itemize}
\item \textbf{First}) We create a \verb~user~ defined schema with composer
\item \textbf{Second}) generate xml-data (\emph{assume correct} )
\item \textbf{Third}) run api command to select schema (I.e title, Id, etc\ldots{})
\item \textbf{Fourth}) build \emph{payload} to be uploaded
\item \textbf{Fifth}) run api command to upload data
\item \textbf{RESULT}) ``NO SCHEMA FOUND WITH THIS ID''
\item \textbf{BUG}) even after hardcoding the selected schema Id upload fails
\item \textbf{Thoughts}) I believe that this due to the \verb~User~ defined nature of the
 schema
\end{itemize}
\subsubsection{Promoting User Defined schema to Global schema}
\label{sec-1-2-3}

\begin{itemize}
\item \textbf{First}) USER A creates a \verb~user~ defined schema with composer using complex types
\item \textbf{Second}) USER B wishes to utilize schemas developed by USER A
\item \textbf{Bug}) USER B \underline{cannot} \emph{see} USER A developed schemas
\item \textbf{Note}) Does this affect the api? (\emph{assuming the api works})
\end{itemize}
\subsubsection{Complex-simple type collisions \emph{for buckets}}
\label{sec-1-2-4}

\begin{itemize}
\item \textbf{First}) create \verb~complex~ types in oxygen
\item \textbf{Second}) upload \verb~complex types~ using the dash board
\item \textbf{Third}) Realized that 2 or more complex types contain identically named
\end{itemize}
simple types \textbf{but} in different context
\begin{itemize}
\item \textbf{Fourth}) build \verb~user~ defined schema in composer
\item \textbf{Bug}) cannot create template because of propert stated in \textbf{Third} point
\end{itemize}
we believe that a collision of type names occurs.
\subsection{Concerns}
\label{sec-1-3}

   
\subsubsection{Production Level Server subdirectory}
\label{sec-1-3-1}

    Currently we are trying to deploy a production level server but I have a 
    couple of intial concerns
    
\begin{itemize}
\item We have one server. on this server we intend to host \verb~multiple~ web/network
\end{itemize}
    based applications for materials data. So we will be presented with the 
    following instance:
    
    --Servername: materials.mat.tamu.edu
    
    so we want have a setup like the following 

    --Website 1 --> materials.mat.tamu.edu/homepage

    --Website 2 --> materials.mat.tamu.edu/gitlab

    --Website 3 --> materials.mat.tamu.edu/MSGalaxy

    --Website 4 --> materials.mat.tamu.edu/MDCS

    so the problem is that there is not .ini or .cfg file that would change all 
    the hardcoded html ref=''/'' to ref=''\//MDCS/'' as a subdirectory hosted on
    an apache instance. same is true for redirects

    we were able to use ``sed'' command to replace some of the them and acheived 
    some function of mdcs and begun to play around with a configuration program
    of our own.
\subsubsection{Leaving debug Mode}
\label{sec-1-3-2}


    Inside of the settings.py there are different means of deploying Django 
    depending on if you are debuging or on production level.

    One notable difference is the implementation of HTTPS and postgres. I was
    able to make some changes. But a standard git branch or set of instructions
    will have to be addresed at some point.
      

\end{document}
