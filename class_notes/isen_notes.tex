% Created 2015-12-09 Wed 15:17
\documentclass[11pt]{article}
\usepackage[utf8]{inputenc}
\usepackage[T1]{fontenc}
\usepackage{fixltx2e}
\usepackage{graphicx}
\usepackage{grffile}
\usepackage{longtable}
\usepackage{wrapfig}
\usepackage{rotating}
\usepackage[normalem]{ulem}
\usepackage{amsmath}
\usepackage{textcomp}
\usepackage{amssymb}
\usepackage{capt-of}
\usepackage{hyperref}
\author{daniel}
\date{\today}
\title{}
\hypersetup{
 pdfauthor={daniel},
 pdftitle={},
 pdfkeywords={},
 pdfsubject={},
 pdfcreator={Emacs 24.3.1 (Org mode 8.3.2)}, 
 pdflang={English}}
\begin{document}

\tableofcontents

\section{FINAL}
\label{sec:orgheadline21}
\subsection{CHAPTER 1}
\label{sec:orgheadline1}
\begin{itemize}
\item Time Value of Money
\item Cash Flow Diagrams
\begin{itemize}
\item Draw a CF diagram
\end{itemize}
\end{itemize}
\subsection{CHAPTER 2}
\label{sec:orgheadline2}
\begin{itemize}
\item Recognition and conversion of cash flows
\item Shifted Series
\end{itemize}
\subsection{CHAPTER 3}
\label{sec:orgheadline3}
\subsection{CHAPTER 4 \& 5}
\label{sec:orgheadline6}
\subsubsection{TYPE of Alternatives}
\label{sec:orgheadline4}
single Project
If PW/AW/FW at MARR >= 0, \textbf{ACCEPT}

\subsubsection{SOme Definintions (VERY IMPORTANT}
\label{sec:orgheadline5}
\begin{itemize}
\item Capitalized worth
\begin{equation}
 (CW)-  CW= \frac{AW}{i}
\end{equation}
\end{itemize}
\begin{itemize}
\item COST (CC)-

\item Present worth of \(\infty\) lives
\end{itemize}

\subsection{CHAPTER 6: Rates of Return (going to have to read)}
\label{sec:orgheadline8}

\begin{itemize}
\item \textbf{Internal Rate of Return} (IRR) is \uline{not} always unique.
\item \textbf{External Rate of Return} (ERR) is \uline{always} always unique.

These two are not ranking methods.   ??? What does that mean

When analyzing several alternatives, and incremental approach is applied

Conclusions from PW/AW/FW/B/C GIVE The \uline{SAME} result.
\end{itemize}

\subsubsection{INcremental IRR/ERR Analysis}
\label{sec:orgheadline7}
These are the steps required to do an IRR/ERR Analysis

\begin{enumerate}
\item Order the alternatives (two alternatives) by the \uline{initial} investment

\item Develop the incremental cash flow series. Draw an \uline{icremental} cash flow diagram, if needed.
\end{enumerate}


??? what is an incremental cash flow diagram

\begin{enumerate}
\item Find the Incremenal rate of reutn for this series, term it \(\Delta\) i\(^{\text{*}}\)

\item Select the better one
\begin{itemize}
\item If \(\Delta\) i\(^{\text{*}}_{\text{B-A}}\) < MARR then select A
\item If \(\Delta\) i\(^{\text{*}}_{\text{B-A}}\) >= MARR then select B
\end{itemize}
\end{enumerate}

\subsection{CHAPTER 8: Depreciation}
\label{sec:orgheadline13}
\subsubsection{Concepts}
\label{sec:orgheadline9}
\begin{itemize}
\item Depreciation in \uline{not} cashflow
\item it \uline{is} Deducted from teh taxable income
\item Decreases tax magnitud (Ch.9)
\end{itemize}

\subsubsection{Depreciation Methods}
\label{sec:orgheadline10}
\begin{itemize}
\item Straight Line SL (THE SAME AMOUNT EVERY YEAR)
\item DECLING BALANCE (DB): accelerated write-off
\begin{equation}
\item MACRS----- P= \frac{2}{n}
      \end{equation}
\end{itemize}

\subsubsection{Summary of Depreciation Method Relations}
\label{sec:orgheadline11}

\begin{center}
\begin{tabular}{llll}
\hline
Method & SL & DB & MACRS\\
\hline
 &  &  & \\
 &  &  & \\
Depreciation rate p & \[ \begin{equation} $\frac{1}{n} $ \end{equation} \] & 1/n, 1.5/n, & Varies per\\
 & \[ $ \frac{1}{n} $ \] & 1/n, 1.5/n, & Varies per\\
 &  & 2/n & year\\
\hline
 &  &  & \\
Annual depreciation d\(_{\text{t}}\) & (P-F)p= & (B\(_{\text{t-1}}\))p= & P\(_{\text{t}}\)*P\\
 & (P - F) / n & $\backslash$=P(1-p)\(^{\text{t-1}}\)p & \\
\hline
 &  & B\(_{\text{t-1}}\)-d\(_{\text{t}}\)= & B\(_{\text{t-1}}\)-d\(_{\text{t}}\)\\
Book value B\(_{\text{t}}\) & B\(_{\text{t-1}}\)-d\(_{\text{t}}\) & P(1 - p)\(^{\text{t}}\) & \\
\hline
\end{tabular}
\end{center}

\begin{itemize}
\item \textbf{SL:} BV dcreases by a \uline{constant amount} annually.
\item \textbf{DB:} BV dcreases by a \uline{constant percentage} annually.
\item \textbf{MACRS:} BV dcreases by a \uline{various percentage} annually.
\end{itemize}

\subsubsection{WORK OUT EXAMPLES}
\label{sec:orgheadline12}
TODO
\subsection{CHAPTER 9}
\label{sec:orgheadline16}
\subsubsection{Corporate Income Tax Rates}
\label{sec:orgheadline14}
\begin{itemize}
\item VIEW TABLE 9.1

\item Effective Tax Rate (or averaged tax rate)
\begin{itemize}
\item The income tax divided by the taxable income.
\end{itemize}
\item Incremental Tax Rate
\begin{itemize}
\item The incremental income tax divided by the incremental investment.
\end{itemize}
\item Marginal Tax Rate
\begin{itemize}
\item The tax rate that will apply to the last dollar included in taxable income.
\end{itemize}
\end{itemize}

EXAMPLES

\subsubsection{BTCF and ATCF Analysis}
\label{sec:orgheadline15}
\begin{itemize}
\item Before Tax Cash Flow --- BTCF:
\begin{itemize}
\item \uline{ALL} the cash flows \uline{except} \texttt{taxes} and \texttt{loan} payment
\end{itemize}
\item After Tax Cash Flow --- ATCF:
+ATCF = BTCF -Tax - Loan Payment
\end{itemize}

Loan Payment = Principal Payment + Interest Payment

Tax= Taxable Income * (tax-rate)

Taxable Income =
= BTCF -Depreciation - Loan Interest Payment

\subsection{CHAPTER 11: Break Even Analysis}
\label{sec:orgheadline20}
\subsubsection{Break-Even Analysis}
\label{sec:orgheadline17}
\begin{itemize}
\item Break-Even Analysis
\begin{itemize}
\item A method ?????
\end{itemize}
\item Break-Even Value
\begin{itemize}
\item The value of a parameter at whoch the measure of economic worth equates to zero
????
\end{itemize}
\item EXAMPLE
\end{itemize}

\subsubsection{Sensitivity Analysis}
\label{sec:orgheadline18}
\begin{itemize}
\item A method used to determine the impact on the measure of economic worth when values of \texttt{one or more parameters} vary over specified ranges.
\end{itemize}
\subsubsection{Break-Even Analysis}
\label{sec:orgheadline19}
\begin{itemize}
\item incorporates explicityly random variation in one or more parameters

\item Then  finds a risk measure (PW/AW/FW)

\item Or/and, finds the probaility of economic worth to be greater than 0 ( or IRR/ERR > MARR)
\end{itemize}
\end{document}
